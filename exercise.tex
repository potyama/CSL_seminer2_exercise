\documentclass[a4paper,12pt]{article}

\usepackage{amsmath}
\usepackage{amssymb}
\usepackage{graphicx}
\usepackage{booktabs}
\usepackage{hyperref}

\title{LaTeX演習問題}
\author{あなたの名前}
\date{\today}

\begin{document}

\maketitle
\tableofcontents

\newpage

\section{イントロダクション}

この演習問題は、LaTeXの基本的な機能に慣れるためのものです。各セクションに示された演習を解きながら、LaTeXの使い方をマスターしていきましょう。

\section{数式に関する演習}

次の数式をLaTeXを使って表現してください。

\subsection{演習1: インライン数式}
\begin{enumerate}
\item 

\item 
\end{enumerate}

\subsection{演習2: ディスプレイ数式}

次の数式をディスプレイ数式として書いてください。
\begin{enumerate}
    \item 二次方程式の解の公式:


    \item 定積分

\end{enumerate}

\subsection{演習3: 番号つきディスプレイ数式}
\begin{enumerate}
\item 以下の数式を番号つきのディスプレイ数式で記述してください:


\item 以下の数式を記述し、番号つきのディスプレイ数式で記述してください:

\end{enumerate}

\section{箇条書きに関する演習}

以下の箇条書きをLaTeXで表現してください。

\subsection{演習4: 箇条書き}

フルーツのリストを箇条書き形式で表示してください。

\subsection{演習5: 番号付きリスト}

以下のように番号付きリストで表現してください。


\section{図の挿入に関する演習}

\subsection{演習6: 図の挿入}

演習ファイルに図を挿入してください。ファイル名は `sample.eps2` で、imgフォルダに用意されています。また、幅を `5cm` に設定してください。

\section{参考文献に関する演習}

\subsection{演習7: 文献引用}
以下の文章のように引用を追加してください。

樋口らは、eBPFを用いたリアルタイム防御システムを考案した。

\begin{thebibliography}{1}
    \bibitem{Higuchi2023}K. Higuchi and R. Kobayashi, "Real-Time Defense System using eBPF for Machine Learning-Based Ransomware Detection Method," 2023 Eleventh International Symposium on Computing and Networking Workshops (CANDARW), 2023, pp. 213-219, doi: 10.1109/CANDARW60564.2023.00043. 
\end{thebibliography}
\end{document}
